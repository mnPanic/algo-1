\documentclass[a4paper]{article}
    \input{Algo1Macros}
    
    \usepackage{a4wide}
    \usepackage{amsmath, amscd, amssymb, amsthm, latexsym}
    \usepackage[spanish,activeacute]{babel}
    \usepackage{enumerate}
    
    \setlength{\parskip}{0.1em}
    \usepackage{caratula} % Version modificada para usar las macros de algo1 de ~> https://github.com/bcardiff/dc-tex
    
    % -------- MACROS ----------
    
    % Define una tupla de enteros
    \newcommand{\tEnt}{\ent \times \ent}
    \newcommand{\bt}{\breakAndSpace{\tab}}
    \newcommand{\btt}{\breakAndSpace{6em}}
    \newcommand{\bttt}{\breakAndSpace{8em}}
    \newcommand{\bqt}{\breakAndSpace{1em}}
    
    \newcommand{\forAll}[2]{(\forall{\ #1: #2})\ }
    \newcommand{\Existe}[2]{(\exists{\ #1: #2})\ }
    
    % -------- END MACROS ------
    
    \begin{document}
    
    \titulo{TP de Especificación}
    \subtitulo{Juego de la vida toroidal}
    \fecha{\today}
    \materia{Algoritmos y Estructuras de Datos I}
    \grupo{Grupo: FrankerZ}
    
    % Pongan cuantos integrantes quieran
    \integrante{Manuel Panichelli}{072/18}{panicmanu@gmail.com}
    \integrante{Ignacio Alonso Rehor}{195/18}{arehor.ignacio@gmail.com}
    
    \maketitle
    
    \section{Problemas}
    %\aux{Aux}{i: \ent}{\bool}{\True}
    type $toroide = \matriz{\bool}$
    
    \subsection{esValido}
    % Ejercicio 1 : proc esValido(in t: toroide, out result : Bool)
    % Que dado un toroide verifique si es válido
    
    % Un toroide será válido si es una Matriz valida y no es vacío
    
    \begin{proc}{esValido}{
            \In t: $toroide$,
            \Out result: $\bool$}{}
        \pre{\True}
        \post{result = true \Iff esToroideValido(t)}
    \end{proc}
    
    \subsection{posicionesVivas}
    % Ejercicio 2 : proc posicionesVivas(in t: toroide, out vivas : seq<Z × Zi>)
    % Que dado un toroide devuelva todas las posiciones vivas.
    \begin{proc}{posicionesVivas}{
            \In t: $toroide$,
            \Out vivas: \TLista{\tEnt}}{}
        \pre{esToroideValido(t)}
        \post{
            (sonPosicionesValidas(vivas,\ t) \yLuego \bqt
            \ sonPosicionesVivas(vivas,\ t)) \wedge \bqt
            \ (cantidadDePosicionesVivas(t) = \longitud{vivas})}
    \end{proc}
    
    \pred{sonPosicionesValidas}{ps: \TLista{\tEnt}, t: $toroide$}{
        \forAll{p}{\tEnt} (p \in ps \implicaLuego esPosicionValida(t, p))
    }
    
    \pred{sonPosicionesVivas}{vivas: \TLista{\tEnt}, t: $toroide$}{
        \forAll{p}{\tEnt} (p \in vivas \implicaLuego esPosicionViva(t, p))
    }
    
    \subsection{densidadPoblacion}
    % Ejercicio 3 : Que dado un toroide devuelva su densidad de poblaci ́on, es decir, la relaci ́on entre la cantidad de posiciones vivas y la cantidad total de posiciones.
    
    \begin{proc}{densidadPoblacion}{
            \In t: $toroide$, 
            \Out result: \float}{}
        \pre{esToroideValido(t)}
        \post{result \times cantidadTotalDePosiciones(t) = cantidadDePosicionesVivas(t)}
    \end{proc}
    
    \aux{cantidadTotalDePosiciones}{t: $toroide$}{\ent}{cols(t) \times rows(t)}
    
    \subsection{evolucionDePosicion}
    % Ejercicio 4 : proc evolucionDePosicion(in t: toroide, in posicion : Z × Z, out result : Bool)
    % Que dado un toroide y una posición del mismo, indique si dicha posici´on estar´ıa viva luego de un tick.
    \begin{proc}{evolucionDePosicion}{
            \In t: $toroide$,
            \In p: \tEnt,
            \Out result: \bool}{}
        \pre{esToroideValido(t) \wedge esPosicionValida(t, p)}
        \post{result = true \Iff viveLuegoDeUnTick(t, p)}
    \end{proc}
    
    
    \subsection{evolucionToroide}
    % Ejercicio 5 : proc evolucionToroide(inout t: toroide) Que dado un toroide lo evolucione un tick.
    \begin{proc}{evolucionToroide}{
            \Inout t: $toroide$}{}
        \pre{esToroideValido(t) \wedge t =$ T_{0}$}
        \post{esToroideValido(t) \wedge esEvolucion(t, T_{0})}
    \end{proc}
    
    \subsection{evolucionMultiple}
    % Ejercicio 6 : proc evolucionMultiple(in t: toroide, in k: Z, out result: toroide)
    % Que dado un toroide t y un natural k, devuelva el toroide resultante de evolucionar t por k ticks.
    
    \begin{proc}{evolucionMultiple}{
            \In t : $toroide$, 
            \In k : \ent, 
            \Out result : $toroide$}{}
        \pre{(k \geq 1) \wedge
             esToroideValido(t)}
        \post{esK-EsimaEvolucion(result, t, k)}
    \end{proc}
    
    \subsection{esPeriodico}
    %Ejercicio 7: proc esPeriodico(in t: toroide, inout p: Z, out result: Bool)
    %Que dado un toroide devuelva si el mismo es peri ́odico o no. En caso de serlo, se debe devolver en la m ́ınima
    %cantidad de ticks en la cual se repite el patr ́on. Decimos que un toroide es peri ́odico si pasada cierta cantidad de ticks,
    % vuelve a tener exactamente la misma configuraci ́on que ten ́ıa originalmente.
    %Por ejemplo, en el siguiente caso el toroide es peri ́odico, y p = 2.
    
    \begin{proc}{esPeriodico}{
            \In t : $toroide$, 
            \Inout p: \ent,
            \Out result : \bool}{}
        \pre{esToroideValido(t)}
        \post{(result = \True \Iff tienePeriodicidad(t)) \wedge 
              esMinimaPeriodicidad(t, p)}
    \end{proc}
    
    \subsection{primosLejanos}
    % Ejercicio 8: proc primosLejanos(in t1: toroide, in t2: toroide, out primos: Bool)
    %Que dados dos toroides, devuelva si uno es la evoluci ́on m ́ultiple del otro.
    
    \begin{proc}{primosLejanos}{
            \In t1 : $toroide$, 
            \In t2 : $toroide$, 
            \Out primos : \bool}{}
        \pre{esToroideValido(t1) \wedge esToroideValido(t2)}
        \post{primos = \True \Iff sonPrimosLejanos(t1, t2)}
    \end{proc}
    
    % si son primos lejanos quiere decir que 
    % - t1 es alguna evolución de t2 o al revès
    % existe algun índice tal que t1 es K-esima evolución 
    \pred{sonPrimosLejanos}{t1: $toroide$, t2: $toroide$}{
        \Existe{k}{\ent}((k \geq 0) \yLuego
            ($esK-EsimaEvolucion(t1, t2, k)$ \ \vee \
             $esK-EsimaEvolucion(t2, t1, k)$))
    }
    
    
    \subsection{seleccionNatural}
    % Ejercicio 9 : proc seleccionNatural(in ts: seqhtoroidei, out res: Z)
    % Que dada una secuencia de toroides, devuelva el ´ındice de aquel toroide que m´as ticks tardar´a en morir. Se considera que un toroide muere cuando no tiene posiciones vivas.
    
    \begin{proc}{seleccionNatural}{
            \In ts: \TLista{toroide},
            \Out res: \ent}{}
        \pre{sonToroidesValidos(ts)}
        \post{(0 \leq \ $res$ \ < \longitud{ts}) \yLuego \bqt
              ($noMuere(ts[res])$\ \vee \
               $esElDeMuerteMasTardia(ts[res], ts)$)}
    \end{proc}
    
    \pred{noMuere}{t: toroide}{
        tienePeriodicidad(t)
    }
    
    \pred{esElDeMuerteMasTardia}{t: toroide, ts: \TLista{toroide}}{
        \forAll{w}{toroide} (w \in ts\implicaLuego muereDespues(t, w))
    }
    
    \pred{muereDespues}{t, w: toroide}{
        \Existe{n, m}{\ent}((n, m \geq 0) \yLuego (muereEnTick(t, n) \wedge muereEnTick(w, m) \wedge (n \geq m)))
    }
    
    \begin{verbatim}
    /* Un toroide t muere en k ticks si en su k-esima evolución esta muerto
     y en todas las anteriores esta vivo */
    \end{verbatim}
    
    \pred{muereEnTick}{t: toroide, k: \ent}{
        \Existe{t_{k}}{toroide}
            ($esK-EsimaEvolucion$(t_{k}, t, k)\ \wedge \ 
             $estaMuerto(t_{k})$) \wedge \bt
        \forAll{q}{\ent}
            ((0\leq q < k) \implicaLuego 
                (\Existe{t_{q}}{toroide} 
                    ($esK-EsimaEvolucion(t_{q}, t, q)$ \wedge \ 
                     \neg \ $estaMuerto(t_{k}$)))
    }
    
    \pred{estaMuerto}{t: toroide}{
        \forAll{p}{\tEnt} 
            (esPosicionValida(t, p) \implicaLuego \neg \ esPosicionViva(t, p))
    }
    
    \subsection{fusionar}
    % Ejercicio 10 : proc fusionar(in t1: toroide, in t2: toroide, out res: toroide)
    % Que dados dos toroides de la misma dimensi´on, devuelva otro (de la misma dimensi´on) que tenga vivas solo aquellas posiciones que estaban vivas en ambos toroides.
    
    \begin{proc}{fusionar}{
            \In t1: toroide,
            \In t2: toroide,
            \Out res: toroide}{}
        \pre{(esToroideValido(t1) \wedge 
             esToroideValido(t2)) \yLuego \bqt
             mismaDimension(t1, t2)}
        \post{(esToroideValido(res) \yLuego \bqt 
              mismaDimension(t1, res)) \yLuego \bqt
              compartenPosicionesVivas(t1, t2, res)}
    \end{proc}
    
    \begin{verbatim}
    /* Supone que t1, t2, y t_{r} tienen la misma dimension. */
    \end{verbatim}
    
    \pred{compartenPosicionesVivas}{t1, t2, tr}{
        \forAll{p}{\tEnt} (esPosicionValida(t1, p) \implicaLuego \bt 
            (esPosicionViva(res, p) \Iff (esPosicionViva(t1, p) \wedge esPosicionViva(t2, p))))
    }
    
    \subsection{vistaTrasladada}
    %Ejercicio 11 : proc vistaTrasladada(in t1: toroide, in t2: toroide, out res: Bool)
    %Que dados dos toroides de la misma dimensi´on, indica si uno es el resultado de trasladar la vista en el otro. Es decir, que
    %moviendo el centro del eje de coordenadas de uno de los toroides en alguna direcci´on, se obtiene el otro. Por ejemplo, los
    %toroides que se ven a continuaci´on cumplen esta propiedad, ya que si se aplica un desplazamiento de 4 lugares a la derecha
    %en el eje X y de 3 lugares hacia abajo en el eje Y al toroide de la izquierda, se obtiene el de la derecha:
    
    
    \begin{proc}{vistaTrasladada}{
            \In t1: toroide,
            \In t2: toroide,
            \Out res: \bool}{}
        \pre{(esToroideValido(t1) \wedge 
             esToroideValido(t2)) \yLuego \bqt
             mismaDimension(t1, t2)}
        \post{res = \True \Iff esTraslacion(t1,t2)}
    \end{proc}
    
    \subsection{enCrecimiento}
    % Ejercicio 12 : proc enCrecimiento(in t: toroide, out res: Bool)
    % Que verifica si la menor superficie que cubre a todas las celdas vivas se va incrementar en el pr´oximo tick
    
    \begin{proc}{enCrecimiento}{
            \In t: toroide,
            \Out res: \bool}{}
        \pre{esToroideValido(t)}
        \post{res = \True \Iff areaIncrementa(t)}
    \end{proc}
    
    
    \pred{areaIncrementa}{t: toroide}{
        // existe t_1 toroide evolucionado
        // existen a, b Z (esMinimaAreaQueCubre(t, a) y esMinimaA...(t_1, b) y a<b
    }
    
    \pred{esMinimaAreaQueCubre}{t: toroide, a: \ent}{
        // a es el area minima de t si a es area de t y aparte para cualquier otro area de t a' a <= a'
    }
    
    \pred{esAreaQueCubre}{t: toroide, a: \ent}{
        // area(t) <= a
        // existe una direccion d tal que existe una matriz m que cubre a las posiciones vivas de t y area(m) = a
    }
    
    \aux{area}{m: \matriz{T}}{\ent}{
        cols(m) \times rows(m)
    }
    
    \pred{cubrePosicionesVivas}{t: toroide, m: \matriz{T}}{
        // paratoda p en tent p es posicion valida de t implica luego (es posicion viva de t & 
        
        para toda posiicon viva, su posicion relativa en la matriz tiene que estar en rango.
    }
    
    \pred{estaEnRango}{p: \tEnt, m: \matriz{T}}{
        0 <= p_{0} < cols(m)
        0 <= p_{1} < rows (m)
    }
    
    \aux{posicionRelativa}{p: \tEnt, r: \tEnt}{\tEnt}{
        le resta r a p
    }
    
    \pred{esRValido}{r: \tEnt, m: \matriz{T}, t: toroide}{
        //  0 <= r_{0} + cols(m) <= cols(t)
        //  0 <= r_{1} + rows(m) <= rows(t)
    }
    
    Hay una translacion del toroide que hace que la superficie que cubre a todos sea mínima (por las particularidades de los toroides)
    
    
    
    
    \section{Predicados y Auxiliares generales}
    
    \subsection{Toroides}
    
    \pred{esToroideValido}{t: $toroide$}{
        esMatriz(t) \wedge \bt
        (rows(t) \geq 3) \wedge \bt
        (cols(t) \geq 3)
    }
    
    \pred{sonToroidesValidos}{ts: \TLista{$toroide$}}{
        \forAll{t}{toroide} (t \in ts \implicaLuego esToroideValido(t))
    }
    
    
    %/* Supone que es una posicion válida (no se va a ir de rango) /*
    \aux{valorEn}{t: $toroide$, p: \tEnt}{\bool}{
        t[p_{0}][p_{1}]
    }
    
    \pred{esPosicionViva}{t: $toroide$, p: \tEnt}{
        % Chequea que t[i,j] sea una pos viva.
        (valorEn(t, p) = \True)
    }
    
    \aux{cantidadDePosicionesVivas}{t: $toroide$}{\ent} {
        cantidadDeAparicionesEnMat(True, t)
    }
    
    \pred{esEvolucion}{t, t_{0}: toroide}{
        mismaDimension(t, t_{0}) \yLuego \bt
        \forAll{p}{\tEnt} (esPosicionValida(t, p) \implicaLuego valorEn(t, p) =  viveLuegoDeUnTick(t_{0}, p))
    }
    
    \pred{mismaDimension}{t, t_{0}: toroide}{
        (cols(t) = cols(t_{0})) \wedge \bt
        (rows(t) = rows(t_{0}))
    }
    
    
    \pred{esK-EsimaEvolucion}{t, t_{0}: toroide, k: \ent}{
        \Existe{ts}{\TLista{toroide}} \btt (
            sonToroidesValidos(ts) \yLuego \bttt ( 
                (ts[0] = t_{0}) \wedge
                (ts[k] = t) \wedge \bttt
                \forAll{i}{\ent} (0 \leq i < \longitud{ts} - 1) \implicaLuego 
                    esEvolucion(ts[i + 1], ts[i])))
            )
        )
    }
    
    \pred{tienePeriodicidad}{t: $toroide$}{
        (\exists{k: \ent})($tienePeriodicidadK-Esima(t, k)$)
    }
    
    % p es el minimo si:
    % - t tiene periodicidad p
    % - para todo q tal que t tiene periodicidad q-esima
    % => p <= q
    \pred{esMinimaPeriodicidad}{t: $toroide$, p: \ent}{
        $tienePeriodicidadK-Esima(t, p)$\ \wedge \bt
        \forAll{q}{\ent} ($tienePeriodicidadK-Esima(t, q) $ \implica p \leq q)
    }
    
    \pred{tienePeriodicidadK-Esima}{t: $toroide$, k: \ent}{
        (k \geq 1) \wedge ($esK-EsimaEvolucion(t, t, k)$)
    }
    
    % TODO: Preguntar si la funcion "dada" mod cumple que el resultado es mayor o igual a 0
    \aux{trasladar}{t: $toroide$, p: \tEnt, d: \tEnt}{\tEnt}{
        (mod(p_{0} + d_{0}, cols(t), \bt
         mod(p_{1} + d_{1}, rows(t))
    }
    
    \pred{esTraslacion}{t1, t2: toroide}{
        \Existe{d}{\tEnt}esTraslacionEnDireccion(t1, t2, d)
    }
    \pred{esTraslacionEnDireccion}{t, t_{0}: toroide, d: \tEnt}{
        \forAll{p}{\tEnt}
            (esPosicionValida(t_{0}, p) \implicaLuego
                (valorEn(t_{0}, p) = valorEn(t, trasladar(t_{0}, p, d))))
    }
    
    \subsubsection{Vecinos}
    % - Cualquier posici´on viva con menos de 2 vecinas vivas, muere (por soledad)
    % - Cualquier posici´on viva con 2 o 3 vecinos vivos, vive.
    % - Cualquier posici´on viva con m´as de 3 vecinas vivas, muere (por %superpoblaci´on)
    % - Cualqueir posici´on muerta con exactamente 3 vecinos vivos, pasa a vivir (por reproducci´on)
    
    \pred{viveLuegoDeUnTick}{t: $toroide$, p: \tEnt}{
        (\neg\ esSoledad(t, p) $ \wedge$ \neg\ esSuperpoblacion(t, p)) \wedge \bt
        (esSupervivencia(t, p) \vee esReproduccion(t, p))
    }
    
    \pred{esSoledad}{t: $toroide$, p: \tEnt}{
        esPosicionViva(t, p) \wedge
        (cantidadVecinosVivos(t, p) < 2)
    }
    
    \pred{esSuperpoblacion}{t: $toroide$, p: \tEnt}{
        esPosicionViva(t, p) \wedge
        (cantidadVecinosVivos(t, p) > 3)
    }
    
    \pred{esSupervivencia}{t: $toroide$, p: \tEnt}{
        esPosicionViva(t, p) \wedge
        (2 \leq cantidadVecinosVivos(t, p) \leq 3)
    }
    
    \pred{esReproduccion}{t: $toroide$, p: \tEnt}{
        ($\neg$ esPosicionViva(t, p)) \wedge
        (cantidadVecinosVivos(t, p) = 3)
    }
    
    \aux{cantidadVecinosVivos}{t: $toroide$, p: \tEnt}{\tEnt}{
        \sum_{i=-1}^1 \sum_{j=-1}^1 
        \IfThenElse{(
            \neg\ $esPosicionViva(t, trasladar(t, p, (i, j)) $ 
            \vee\ (i, j) = (0,0))}
            {0}
            {1}
    }
    
    \subsection{Matrices}
    
    %/* Para filas y columnas suponemos que m es una matriz */\\
    \aux{cols}{m: \matriz{T}}{\ent}{\longitud{m}}
    \aux{rows}{m: \matriz{T}}{\ent}{\IfThenElse{cols(m) > 0}{\longitud{m[0]}}{0}}
    
    \pred{esMatriz}{m: \matriz{T}}{
        \paraTodo{i,j}{(0}{\longitud{m}} \implicaLuego \longitud{m[i]} = \longitud{m[j]})
    }
    
    \aux{cantidadDeAparicionesEnMat}{x: T, m: \matriz{T}}{\ent}{
        \sum_{i=0}^{cols(m) - 1}
        \sum_{j=0}^{rows(m) - 1}
        (\IfThenElse{s[i][j] = x}{1}{0})
    }
    
    \pred{esPosicionValida}{m: \matriz{T}, p: \tEnt}{
        % Chequea que ambos estén en rango
        (0 \leq p_{0} < cols(m)) \wedge \bt
        (0 \leq p_{1} < rows(m))
    }
    
    \section{Decisiones tomadas}
    % Sólo para decisiones de alto nivel, no para explicar la solución a los ejercicios: CONSULTAR!
    - Tomamos al Toroide como una Matriz compuesta por una \TLista{\TLista{\bool}}
    Donde la primera representa a las columnas, y la segunda a las filas.
    Accederemos a ella mediante tuplas $(x, y) : \tEnt$, donde $x$ es la posición en la columna e $y$ en la fila.
    
    - Tomamos como decisión que un toroide válido será aquel que sea al menos de $3 \times 3$, de forma tal que los 8 vecinos de cualquier posicion sean siempre diferentes entre sí.
    
    \end{document}
    
    