\documentclass[a4paper]{article}
\input{Algo1Macros}

\usepackage{a4wide}
\usepackage{amsmath, amscd, amssymb, amsthm, latexsym}
\usepackage[spanish,activeacute]{babel}
\usepackage{enumerate}

\setlength{\parskip}{0.1em}
\usepackage{caratula} % Version modificada para usar las macros de algo1 de ~> https://github.com/bcardiff/dc-tex

\begin{document}

\titulo{TP de Especificación}
\subtitulo{Juego de la vida toroidal}
\fecha{\today}
\materia{Algoritmos y Estructuras de Datos I}
\grupo{Grupo: TBD}

\newcommand{\senial}{\textit{se\~nal}}

% Pongan cuantos integrantes quieran
\integrante{Manuel Panichelli}{072/18}{panicmanu@gmail.com}
\integrante{Ignacio Alonso Rehor}{195/18}{arehor.ignacio@gmail.com}

\maketitle

\section{Problemas}
%\aux{Aux}{i: \ent}{\bool}{\True}
type $toroide = \matriz{\bool}$
\subsection{esValido}
% Ejercicio 1 : proc esValido(in t: toroide, out result : Bool)
% Que dado un toroide verifique si es válido

% Un toroide será válido si es una Matriz valida y no es vacío

\begin{proc}
{esValido}{\In t: $toroide$,
            \out result: $\bool$}{}
    \pre{\True}
    \post{result = true \Iff (\longitud{t} \neq 0 \wedge esMatriz(t))}
\end{proc}

% TODO: Mover a generales?
% TODO: Que trucazo, no? ;)
\pred{esMatriz}{m: \matriz{T}}{
    \paraTodo{i,j}{(0}{\longitud{m}} $\implicaLuego$ \longitud{m[i]} = \longitud{m[j]})
}

\subsection{posicionesVivas}
% Ejercicio 2 : proc posicionesVivas(in t: toroide, out vivas : seq<Z × Zi>)
% Que dado un toroide devuelva todas las posiciones vivas.
\begin{proc}
{posicionesVivas}{\In t: $toroide$,
                  \out vivas: \TLista{\ent \times \ent}}{}
    \pre{esValido(t)}
    \post{sonPosicionesVivas(vivas, t) \wedge }
\end{proc}


\section{Predicados y Auxiliares generales}

/* Para filas y columnas suponemos que m es una matriz */\\
\aux{filas}{m: \matriz{T}}{\ent}{\longitud{m}}
\aux{columnas}{m: \matriz{T}}{\ent}{\IfThenElse{filas(m) > 0}{\longitud{m[0]}}{0}}

\aux{cantidadAparicionesEnSeq}{x: T, s: \TLista{T}}{\ent}{
    \sum_{i=0}^{\longitud{s}-1} (\IfThenElse{s[i] = x}{1}{0})
}

\aux{cantidadAparicionesEnMat}{x: T, m: \matriz{T}}{\ent}{
    \sum_{i=0}^{filas(m) - 1} 
    \sum_{j=0}^{columnas(m) - 1}
    (\IfThenElse{s[i][j] = x}{1}{0})
}

\section{Decisiones tomadas}
% Sólo para decisiones de alto nivel, no para explicar la solución a los ejercicios: CONSULTAR!
Tomamos como decisión que un toroide de 1 solo casillero es válido, permitiendo que se tenga a si mismo como vecino.

\end{document}

