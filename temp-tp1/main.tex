\documentclass[a4paper]{article}
    \input{Algo1Macros}
    
    \usepackage{a4wide}
    \usepackage{amsmath, amscd, amssymb, amsthm, latexsym}
    \usepackage[spanish,activeacute]{babel}
    \usepackage{enumerate}
    
    \setlength{\parskip}{0.1em}
    \usepackage{caratula} % Version modificada para usar las macros de algo1 de ~> https://github.com/bcardiff/dc-tex
    
    % -------- MACROS ----------
    
    % Define una tupla de enteros
    \newcommand{\tEnt}{\ent \times \ent}
    
    % -------- END MACROS ------
    
    
    \begin{document}
    
    \titulo{TP de Especificación}
    \subtitulo{Juego de la vida toroidal}
    \fecha{\today}
    \materia{Algoritmos y Estructuras de Datos I}
    \grupo{Grupo: FrankerZ}
    
    % Pongan cuantos integrantes quieran
    \integrante{Manuel Panichelli}{072/18}{panicmanu@gmail.com}
    \integrante{Ignacio Alonso Rehor}{195/18}{arehor.ignacio@gmail.com}
    
    \maketitle
    
    \section{Problemas}
    %\aux{Aux}{i: \ent}{\bool}{\True}
    type $toroide = \matriz{\bool}$
    
    \subsection{esValido}
    % Ejercicio 1 : proc esValido(in t: toroide, out result : Bool)
    % Que dado un toroide verifique si es válido
    
    % Un toroide será válido si es una Matriz valida y no es vacío
    
    \begin{proc}{esValido}{
            \In t: $toroide$,
            \Out result: $\bool$}{}
        \pre{\True}
        \post{result = true \Iff esToroideValido(t)}
    \end{proc}
    
    % TODO: Mover a generales?
    % TODO: Que trucazo, no? ;)
    \pred{esMatriz}{m: \matriz{T}}{
        \paraTodo{i,j}{(0}{\longitud{m}} $\implicaLuego$ \longitud{m[i]} = \longitud{m[j]})
    }
    
    \subsection{posicionesVivas}
    % Ejercicio 2 : proc posicionesVivas(in t: toroide, out vivas : seq<Z × Zi>)
    % Que dado un toroide devuelva todas las posiciones vivas.
    \begin{proc}{posicionesVivas}{
            \In t: $toroide$,
            \Out vivas: \TLista{\tEnt}}{}
        \pre{esToroideValido(t)}
        \post{sonPosicionesVivas(vivas, t) \wedge 
             (cantidadDePosicionesVivas(t) = \longitud{vivas})}
    \end{proc}
    
    \pred{sonPosicionesVivas}{vivas: \TLista{\tEnt}, t: $toroide$}{
        (\forall{p : \tEnt}) (p \in vivas \Then esPosicionViva(t, p))
    }
    
    \pred{esPosicionViva}{t: $toroide$, p: \tEnt}{
        % Chequea que la pos se vayan de rango i,j
        esPosicionValida(t, p) \yLuego
        % Chequea que t[i,j] sea una pos viva.
        (t[p_{0}][p_{1}] = True)
    }
    
    \subsection{densidadPoblacion}
    % Ejercicio 3 : Que dado un toroide devuelva su densidad de poblaci ́on, es decir, la relaci ́on entre la cantidad de posiciones vivas y la cantidad total de posiciones.
    
    \begin{proc}{densidadPoblacion}{
            \In t: $toroide$, 
            \Out result: \float}{}
        \pre{esToroideValido(t)}
        \post{result \times cantidadTotalDePosiciones(t) = cantidadDePosicionesVivas(t)}
    \end{proc}
    
    \aux{cantidadTotalDePosiciones}{t: $toroide$}{\ent}{cols(t) \times rows(t)}
    
    \subsection{evolucionDePosicion}
    % Ejercicio 4 : proc evolucionDePosicion(in t: toroide, in posicion : Z × Z, out result : Bool)
    % Que dado un toroide y una posición del mismo, indique si dicha posici´on estar´ıa viva luego de un tick.
    \begin{proc}{evolucionDePosicion}{
            \In t: $toroide$,
            \In p: \tEnt,
            \Out result: \float}{}
        \pre{esToroideValido(t) \wedge esPosicionValida(t, p)}
        \post{}
    \end{proc}
    
    \aux{cantidadVecinosVivos}{t: $toroide$, p: \tEnt}{}{}
    
    \section{Predicados y Auxiliares generales}
    
    \subsection{Toroides}
    
    \pred{esToroideValido}{t: $toroide$}{
        (esMatriz(t) 
        \wedge (rows(t) \geq 3
        \wedge cols(t) \geq 3))
    }
    
    \aux{cantidadDePosicionesVivas}{t: $toroide$}{\ent} {
        cantidadDeAparicionesEnMat(True, t)
    }
    
    \subsection{Matrices}
    
    %/* Para filas y columnas suponemos que m es una matriz */\\
    \aux{cols}{m: \matriz{T}}{\ent}{\longitud{m}}
    \aux{rows}{m: \matriz{T}}{\ent}{\IfThenElse{cols(m) > 0}{\longitud{m[0]}}{0}}
    
    \aux{cantidadDeAparicionesEnMat}{x: T, m: \matriz{T}}{\ent}{\\\hspace*{4em}
        \sum_{i=0}^{cols(m) - 1}
        \sum_{j=0}^{rows(m) - 1}
        (\IfThenElse{s[i][j] = x}{1}{0})
    }
    
    \pred{esPosicionValida}{m: \matriz{T}, p: \tEnt}{
        % Chequea que ambos estén en rango
        (0 \leq p_{0} < cols(m)) \wedge (0 \leq p_{1} < rows(m))
    }
    
    \section{Decisiones tomadas}
    % Sólo para decisiones de alto nivel, no para explicar la solución a los ejercicios: CONSULTAR!
    - Tomamos al Toroide como una Matriz compuesta por una \TLista{\TLista{\bool}}
    Donde la primera representa a las columnas, y la segunda a las filas.
    Accederemos a ella mediante tuplas $(x, y) : \tEnt$, donde $x$ es la posición en la columna e $y$ en la fila.
    
    - Tomamos como decisión que un toroide válido será aquel que sea al menos de $3 \times 3$, de forma tal que los 8 vecinos de cualquier posicion sean siempre diferentes entre sí.
    
    \end{document}
    
    