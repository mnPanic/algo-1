\documentclass[a4paper]{article}
    \usepackage{ifthen}
\usepackage{amssymb}
\usepackage{multicol}
\usepackage{graphicx}
\usepackage[absolute]{textpos}

\usepackage[noload]{qtree}
%\usepackage{xspace,rotating,calligra,dsfont,ifthen}
\usepackage{xspace,rotating,dsfont,ifthen}
\usepackage[spanish,activeacute]{babel}
\usepackage[utf8]{inputenc}
\usepackage{pgfpages}
\usepackage{pgf,pgfarrows,pgfnodes,pgfautomata,pgfheaps,xspace,dsfont}
\usepackage{listings}
\usepackage{multicol}


\makeatletter

\@ifclassloaded{beamer}{%
  \newcommand{\tocarEspacios}{%
    \addtolength{\leftskip}{4em}%
    \addtolength{\parindent}{-3em}%
  }%
}
{%
  \usepackage[top=1cm,bottom=2cm,left=1cm,right=1cm]{geometry}%
  \usepackage{color}%
  \newcommand{\tocarEspacios}{%
    \addtolength{\leftskip}{5em}%
    \addtolength{\parindent}{-3em}%
  }%
}

\newcommand{\encabezadoDeProc}[4]{%
  % Ponemos la palabrita problema en tt
%  \noindent%
  {\normalfont\bfseries\ttfamily proc}%
  % Ponemos el nombre del problema
  \ %
  {\normalfont\ttfamily #2}%
  \ 
  % Ponemos los parametros
  (#3)%
  \ifthenelse{\equal{#4}{}}{}{%
  \ =\ %
  % Ponemos el nombre del resultado
  {\normalfont\ttfamily #1}%
  % Por ultimo, va el tipo del resultado
  \ : #4}
}

\newcommand{\encabezadoDeTipo}[2]{%
  % Ponemos la palabrita tipo en tt
  {\normalfont\bfseries\ttfamily tipo}%
  % Ponemos el nombre del tipo
  \ %
  {\normalfont\ttfamily #2}%
  \ifthenelse{\equal{#1}{}}{}{$\langle$#1$\rangle$}
}

% Primero definiciones de cosas al estilo title, author, date

\def\materia#1{\gdef\@materia{#1}}
\def\@materia{No especifi\'o la materia}
\def\lamateria{\@materia}

\def\cuatrimestre#1{\gdef\@cuatrimestre{#1}}
\def\@cuatrimestre{No especifi\'o el cuatrimestre}
\def\elcuatrimestre{\@cuatrimestre}

\def\anio#1{\gdef\@anio{#1}}
\def\@anio{No especifi\'o el anio}
\def\elanio{\@anio}

\def\fecha#1{\gdef\@fecha{#1}}
\def\@fecha{\today}
\def\lafecha{\@fecha}

\def\nombre#1{\gdef\@nombre{#1}}
\def\@nombre{No especific'o el nombre}
\def\elnombre{\@nombre}

\def\practicas#1{\gdef\@practica{#1}}
\def\@practica{No especifi\'o el n\'umero de pr\'actica}
\def\lapractica{\@practica}


% Esta macro convierte el numero de cuatrimestre a palabras
\newcommand{\cuatrimestreLindo}{
  \ifthenelse{\equal{\elcuatrimestre}{1}}
  {Primer cuatrimestre}
  {\ifthenelse{\equal{\elcuatrimestre}{2}}
  {Segundo cuatrimestre}
  {Verano}}
}


\newcommand{\depto}{{UBA -- Facultad de Ciencias Exactas y Naturales --
      Departamento de Computaci\'on}}

\newcommand{\titulopractica}{
  \centerline{\depto}
  \vspace{1ex}
  \centerline{{\Large\lamateria}}
  \vspace{0.5ex}
  \centerline{\cuatrimestreLindo de \elanio}
  \vspace{2ex}
  \centerline{{\huge Pr\'actica \lapractica -- \elnombre}}
  \vspace{5ex}
  \arreglarincisos
  \newcounter{ejercicio}
  \newenvironment{ejercicio}{\stepcounter{ejercicio}\textbf{Ejercicio
      \theejercicio}%
    \renewcommand\@currentlabel{\theejercicio}%
  }{\vspace{0.2cm}}
}  


\newcommand{\titulotp}{
  \centerline{\depto}
  \vspace{1ex}
  \centerline{{\Large\lamateria}}
  \vspace{0.5ex}
  \centerline{\cuatrimestreLindo de \elanio}
  \vspace{0.5ex}
  \centerline{\lafecha}
  \vspace{2ex}
  \centerline{{\huge\elnombre}}
  \vspace{5ex}
}


%practicas
\newcommand{\practica}[2]{%
    \title{Pr\'actica #1 \\ #2}
    \author{Algoritmos y Estructuras de Datos I}
    \date{Primer Cuatrimestre 2018}

    \maketitlepractica{#1}{#2}
}

\newcommand \maketitlepractica[2] {%
\begin{center}
\begin{tabular}{r cr}
 \begin{tabular}{c}
{\large\bf\textsf{\ Algoritmos y Estructuras de Datos I\ }}\\ 
Primer Cuatrimestre 2018\\
\title{\normalsize Gu\'ia Pr\'actica #1 \\ \textbf{#2}}\\
\@title
\end{tabular} &
\begin{tabular}{@{} p{1.6cm} @{}}
\includegraphics[width=1.6cm]{logodpt.jpg}
\end{tabular} &
\begin{tabular}{l @{}}
 \emph{Departamento de Computaci\'on} \\
 \emph{Facultad de Ciencias Exactas y Naturales} \\
 \emph{Universidad de Buenos Aires} \\
\end{tabular} 
\end{tabular}
\end{center}

\bigskip
}


% Simbolos varios

\newcommand{\ent}{\ensuremath{\mathds{Z}}}
\newcommand{\float}{\ensuremath{\mathds{R}}}
\newcommand{\bool}{\ensuremath{\mathsf{Bool}}}
\newcommand{\True}{\ensuremath{\mathrm{true}}}
\newcommand{\False}{\ensuremath{\mathrm{false}}}
\newcommand{\Then}{\ensuremath{\rightarrow}}
\newcommand{\Iff}{\ensuremath{\leftrightarrow}}
\newcommand{\implica}{\ensuremath{\longrightarrow}}
\newcommand{\IfThenElse}[3]{\ensuremath{\mathsf{if}\ $#1$\ \mathsf{then}\ $#2$\ \mathsf{else}\ $#3$\ \mathsf{fi}}}
\newcommand{\In}{\textsf{in }}
\newcommand{\Out}{\textsf{out }}
\newcommand{\Inout}{\textsf{inout }}
\newcommand{\yLuego}{\ensuremath{\land _L\ }}
\newcommand{\oLuego}{\ensuremath{\lor _L\ }}
\newcommand{\implicaLuego}{\ensuremath{\implica _L\ }}
\newcommand{\existe}[3]{\ensuremath{(\exists #1:\ent) \ #2 \leq #1 < #3 \ }}
\newcommand{\paraTodo}[3]{\ensuremath{(\forall #1:\ent) \ (#2 \leq #1 < #3) \ }}

% Símbolo para marcar los ejercicios importantes (estrellita)
\newcommand\importante{\raisebox{0.5pt}{\ensuremath{\bigstar}}}


\newcommand{\rango}[2]{[#1\twodots#2]}
\newcommand{\comp}[2]{[\,#1\,|\,#2\,]}

\newcommand{\rangoac}[2]{(#1\twodots#2]}
\newcommand{\rangoca}[2]{[#1\twodots#2)}
\newcommand{\rangoaa}[2]{(#1\twodots#2)}

%ejercicios
\newtheorem{exercise}{Ejercicio}
\newenvironment{ejercicio}[1][]{\begin{exercise}#1\rm}{\end{exercise} \vspace{0.2cm}}
\newenvironment{items}{\begin{enumerate}[a)]}{\end{enumerate}}
\newenvironment{subitems}{\begin{enumerate}[i)]}{\end{enumerate}}
\newcommand{\sugerencia}[1]{\noindent \textbf{Sugerencia:} #1}

\lstnewenvironment{code}{
    \lstset{% general command to set parameter(s)
        language=C++, basicstyle=\small\ttfamily, keywordstyle=\slshape,
        emph=[1]{tipo,usa}, emphstyle={[1]\sffamily\bfseries},
        morekeywords={tint,forn,forsn},
        basewidth={0.47em,0.40em},
        columns=fixed, fontadjust, resetmargins, xrightmargin=5pt, xleftmargin=15pt,
        flexiblecolumns=false, tabsize=2, breaklines, breakatwhitespace=false, extendedchars=true,
        numbers=left, numberstyle=\tiny, stepnumber=1, numbersep=9pt,
        frame=l, framesep=3pt,
    }
   \csname lst@SetFirstLabel\endcsname}
  {\csname lst@SaveFirstLabel\endcsname}


%tipos basicos
\newcommand{\rea}{\ensuremath{\mathsf{Float}}}
\newcommand{\cha}{\ensuremath{\mathsf{Char}}}

\newcommand{\mcd}{\mathrm{mcd}}
\newcommand{\prm}[1]{\ensuremath{\mathsf{prm}(#1)}}
\newcommand{\sgd}[1]{\ensuremath{\mathsf{sgd}(#1)}}

%listas
\newcommand{\TLista}[1]{\ensuremath{seq \langle #1\rangle}}
\newcommand{\lvacia}{\ensuremath{[\ ]}}
\newcommand{\lv}{\ensuremath{[\ ]}}
\newcommand{\longitud}[1]{\ensuremath{|#1|}}
\newcommand{\cons}[1]{\ensuremath{\mathsf{addFirst}}(#1)}
\newcommand{\indice}[1]{\ensuremath{\mathsf{indice}}(#1)}
\newcommand{\conc}[1]{\ensuremath{\mathsf{concat}}(#1)}
\newcommand{\cab}[1]{\ensuremath{\mathsf{head}}(#1)}
\newcommand{\cola}[1]{\ensuremath{\mathsf{tail}}(#1)}
\newcommand{\sub}[1]{\ensuremath{\mathsf{subseq}}(#1)}
\newcommand{\en}[1]{\ensuremath{\mathsf{en}}(#1)}
\newcommand{\cuenta}[2]{\mathsf{cuenta}\ensuremath{(#1, #2)}}
\newcommand{\suma}[1]{\mathsf{suma}(#1)}
\newcommand{\twodots}{\ensuremath{\mathrm{..}}}
\newcommand{\masmas}{\ensuremath{++}}
\newcommand{\matriz}[1]{\TLista{\TLista{#1}}}

% Acumulador
\newcommand{\acum}[1]{\ensuremath{\mathsf{acum}}(#1)}
\newcommand{\acumselec}[3]{\ensuremath{\mathrm{acum}(#1 |  #2, #3)}}

% \selector{variable}{dominio}
\newcommand{\selector}[2]{#1~\ensuremath{\leftarrow}~#2}
\newcommand{\selec}{\ensuremath{\leftarrow}}

\newcommand{\breakAndSpace}[1]{\\\hspace*{#1}}
%MODIFICACIÓN: breakline antes del cuerpo 3
\newcommand{\pred}[3]{%
    {\normalfont\bfseries\ttfamily pred }%
    {\normalfont\ttfamily #1}%
    \ifthenelse{\equal{#2}{}}{}{\ (#2) }%
    \{\breakAndSpace{3.2em}#3\breakAndSpace{1.5em}\}%
    {\normalfont\bfseries\,\par}%
}

\newenvironment{proc}[4][res]{%
  % El parametro 1 (opcional) es el nombre del resultado
  % El parametro 2 es el nombre del problema
  % El parametro 3 son los parametros
  % El parametro 4 es el tipo del resultado
  % Preambulo del ambiente problema
  % Tenemos que definir los comandos requiere, asegura, modifica y aux
  \newcommand{\pre}[2][]{%
    {\normalfont\bfseries\ttfamily Pre}%
    \ifthenelse{\equal{##1}{}}{}{\ {\normalfont\ttfamily ##1} :}\ %
    \{##2\}%
    {\normalfont\bfseries\,\par}%
  }
  \newcommand{\post}[2][]{%
    {\normalfont\bfseries\ttfamily Post}%
    \ifthenelse{\equal{##1}{}}{}{\ {\normalfont\ttfamily ##1} :}\
    \{##2\}%
    {\normalfont\bfseries\,\par}%
  }
  \renewcommand{\aux}[4]{%
    {\normalfont\bfseries\ttfamily aux\ }%
    {\normalfont\ttfamily ##1}%
    \ifthenelse{\equal{##2}{}}{}{\ (##2)}\ : ##3\, = \ensuremath{##4}%
    {\normalfont\bfseries\,;\par}%
  }
  \newcommand{\res}{#1}
  \vspace{1ex}
  \noindent
  \encabezadoDeProc{#1}{#2}{#3}{#4}
  % Abrimos la llave
  \{\par%
  \tocarEspacios
}
% Ahora viene el cierre del ambiente problema
{
  % Cerramos la llave
  \noindent\}
  \vspace{1ex}
}


% Modificacion: Agrega un linebreak
  \newcommand{\aux}[4]{%
    {\normalfont\bfseries\ttfamily aux\ }%
    {\normalfont\ttfamily #1}%
    \ifthenelse{\equal{#2}{}}{}{\ (#2)}\ : #3\, = \breakAndSpace{3.2em}#4%
    {\normalfont\bfseries\,;\par}%
  }


% \newcommand{\pre}[1]{\textsf{pre}\ensuremath{(#1)}}

\newcommand{\procnom}[1]{\textsf{#1}}
\newcommand{\procil}[3]{\textsf{proc #1}\ensuremath{(#2) = #3}}
\newcommand{\procilsinres}[2]{\textsf{proc #1}\ensuremath{(#2)}}
\newcommand{\preil}[2]{\textsf{Pre #1: }\ensuremath{#2}}
\newcommand{\postil}[2]{\textsf{Post #1: }\ensuremath{#2}}
\newcommand{\auxil}[2]{\textsf{aux }\ensuremath{#1 = #2}}
\newcommand{\auxilc}[4]{\textsf{aux }\ensuremath{#1( #2 ): #3 = #4}}
\newcommand{\auxnom}[1]{\textsf{aux }\ensuremath{#1}}
\newcommand{\auxpred}[3]{\textsf{pred }\ensuremath{#1( #2 ) \{ #3 \}}}

\newcommand{\comentario}[1]{{/*\ #1\ */}}

\newcommand{\nom}[1]{\ensuremath{\mathsf{#1}}}


% En las practicas/parciales usamos numeros arabigos para los ejercicios.
% Aca cambiamos los enumerate comunes para que usen letras y numeros
% romanos
\newcommand{\arreglarincisos}{%
  \renewcommand{\theenumi}{\alph{enumi}}
  \renewcommand{\theenumii}{\roman{enumii}}
  \renewcommand{\labelenumi}{\theenumi)}
  \renewcommand{\labelenumii}{\theenumii)}
}



%%%%%%%%%%%%%%%%%%%%%%%%%%%%%% PARCIAL %%%%%%%%%%%%%%%%%%%%%%%%
\let\@xa\expandafter
\newcommand{\tituloparcial}{\centerline{\depto -- \lamateria}
  \centerline{\elnombre -- \lafecha}%
  \setlength{\TPHorizModule}{10mm} % Fija las unidades de textpos
  \setlength{\TPVertModule}{\TPHorizModule} % Fija las unidades de
                                % textpos
  \arreglarincisos
  \newcounter{total}% Este contador va a guardar cuantos incisos hay
                    % en el parcial. Si un ejercicio no tiene incisos,
                    % cuenta como un inciso.
  \newcounter{contgrilla} % Para hacer ciclos
  \newcounter{columnainicial} % Se van a usar para los cline cuando un
  \newcounter{columnafinal}   % ejercicio tenga incisos.
  \newcommand{\primerafila}{}
  \newcommand{\segundafila}{}
  \newcommand{\rayitas}{} % Esto va a guardar los \cline de los
                          % ejercicios con incisos, asi queda mas bonito
  \newcommand{\anchodegrilla}{20} % Es para textpos
  \newcommand{\izquierda}{7} % Estos dos le dicen a textpos donde colocar
  \newcommand{\abajo}{2}     % la grilla
  \newcommand{\anchodecasilla}{0.4cm}
  \setcounter{columnainicial}{1}
  \setcounter{total}{0}
  \newcounter{ejercicio}
  \setcounter{ejercicio}{0}
  \renewenvironment{ejercicio}[1]
  {%
    \stepcounter{ejercicio}\textbf{\noindent Ejercicio \theejercicio. [##1
      puntos]}% Formato
    \renewcommand\@currentlabel{\theejercicio}% Esto es para las
                                % referencias
    \newcommand{\invariante}[2]{%
      {\normalfont\bfseries\ttfamily invariante}%
      \ ####1\hspace{1em}####2%
    }%
    \newcommand{\Proc}[5][result]{
      \encabezadoDeProc{####1}{####2}{####3}{####4}\hspace{1em}####5}%
  }% Aca se termina el principio del ejercicio
  {% Ahora viene el final
    % Esto suma la cantidad de incisos o 1 si no hubo ninguno
    \ifthenelse{\equal{\value{enumi}}{0}}
    {\addtocounter{total}{1}}
    {\addtocounter{total}{\value{enumi}}}
    \ifthenelse{\equal{\value{ejercicio}}{1}}{}
    {
      \g@addto@macro\primerafila{&} % Si no estoy en el primer ej.
      \g@addto@macro\segundafila{&}
    }
    \ifthenelse{\equal{\value{enumi}}{0}}
    {% No tiene incisos
      \g@addto@macro\primerafila{\multicolumn{1}{|c|}}
      \bgroup% avoid overwriting somebody else's value of \tmp@a
      \protected@edef\tmp@a{\theejercicio}% expand as far as we can
      \@xa\g@addto@macro\@xa\primerafila\@xa{\tmp@a}%
      \egroup% restore old value of \tmp@a, effect of \g@addto.. is
      
      \stepcounter{columnainicial}
    }
    {% Tiene incisos
      % Primero ponemos el encabezado
      \g@addto@macro\primerafila{\multicolumn}% Ahora el numero de items
      \bgroup% avoid overwriting somebody else's value of \tmp@a
      \protected@edef\tmp@a{\arabic{enumi}}% expand as far as we can
      \@xa\g@addto@macro\@xa\primerafila\@xa{\tmp@a}%
      \egroup% restore old value of \tmp@a, effect of \g@addto.. is
      % global 
      % Ahora el formato
      \g@addto@macro\primerafila{{|c|}}%
      % Ahora el numero de ejercicio
      \bgroup% avoid overwriting somebody else's value of \tmp@a
      \protected@edef\tmp@a{\theejercicio}% expand as far as we can
      \@xa\g@addto@macro\@xa\primerafila\@xa{\tmp@a}%
      \egroup% restore old value of \tmp@a, effect of \g@addto.. is
      % global 
      % Ahora armamos la segunda fila
      \g@addto@macro\segundafila{\multicolumn{1}{|c|}{a}}%
      \setcounter{contgrilla}{1}
      \whiledo{\value{contgrilla}<\value{enumi}}
      {%
        \stepcounter{contgrilla}
        \g@addto@macro\segundafila{&\multicolumn{1}{|c|}}
        \bgroup% avoid overwriting somebody else's value of \tmp@a
        \protected@edef\tmp@a{\alph{contgrilla}}% expand as far as we can
        \@xa\g@addto@macro\@xa\segundafila\@xa{\tmp@a}%
        \egroup% restore old value of \tmp@a, effect of \g@addto.. is
        % global 
      }
      % Ahora armo las rayitas
      \setcounter{columnafinal}{\value{columnainicial}}
      \addtocounter{columnafinal}{-1}
      \addtocounter{columnafinal}{\value{enumi}}
      \bgroup% avoid overwriting somebody else's value of \tmp@a
      \protected@edef\tmp@a{\noexpand\cline{%
          \thecolumnainicial-\thecolumnafinal}}%
      \@xa\g@addto@macro\@xa\rayitas\@xa{\tmp@a}%
      \egroup% restore old value of \tmp@a, effect of \g@addto.. is
      \setcounter{columnainicial}{\value{columnafinal}}
      \stepcounter{columnainicial}
    }
    \setcounter{enumi}{0}%
    \vspace{0.2cm}%
  }%
  \newcommand{\tercerafila}{}
  \newcommand{\armartercerafila}{
    \setcounter{contgrilla}{1}
    \whiledo{\value{contgrilla}<\value{total}}
    {\stepcounter{contgrilla}\g@addto@macro\tercerafila{&}}
  }
  \newcommand{\grilla}{%
    \g@addto@macro\primerafila{&\textbf{TOTAL}}
    \g@addto@macro\segundafila{&}
    \g@addto@macro\tercerafila{&}
    \armartercerafila
    \ifthenelse{\equal{\value{total}}{\value{ejercicio}}}
    {% No hubo incisos
      \begin{textblock}{\anchodegrilla}(\izquierda,\abajo)
        \begin{tabular}{|*{\value{total}}{p{\anchodecasilla}|}c|}
          \hline
          \primerafila\\
          \hline
          \tercerafila\\
          \tercerafila\\
          \hline
        \end{tabular}
      \end{textblock}
    }
    {% Hubo incisos
      \begin{textblock}{\anchodegrilla}(\izquierda,\abajo)
        \begin{tabular}{|*{\value{total}}{p{\anchodecasilla}|}c|}
          \hline
          \primerafila\\
          \rayitas
          \segundafila\\
          \hline
          \tercerafila\\
          \tercerafila\\
          \hline
        \end{tabular}
      \end{textblock}
    }
  }%
  \vspace{0.4cm}
  \textbf{Nro. de orden:}
  
  \textbf{LU:}
  
  \textbf{Apellidos:}
  
  \textbf{Nombres:}
  \vspace{0.5cm}
}



% AMBIENTE CONSIGNAS
% Se usa en el TP para ir agregando las cosas que tienen que resolver
% los alumnos.
% Dentro del ambiente hay que usar \item para cada consigna

\newcounter{consigna}
\setcounter{consigna}{0}

\newenvironment{consignas}{%
  \newcommand{\consigna}{\stepcounter{consigna}\textbf{\theconsigna.}}%
  \renewcommand{\ejercicio}[1]{\item ##1 }
  \renewcommand{\proc}[5][result]{\item
    \encabezadoDeProc{##1}{##2}{##3}{##4}\hspace{1em}##5}%
  \newcommand{\invariante}[2]{\item%
    {\normalfont\bfseries\ttfamily invariante}%
    \ ##1\hspace{1em}##2%
  }
  \renewcommand{\aux}[4]{\item%
    {\normalfont\bfseries\ttfamily aux\ }%
    {\normalfont\ttfamily ##1}%
    \ifthenelse{\equal{##2}{}}{}{\ (##2)}\ : ##3 \hspace{1em}##4%
  }
  % Comienza la lista de consignas
  \begin{list}{\consigna}{%
      \setlength{\itemsep}{0.5em}%
      \setlength{\parsep}{0cm}%
    }
}%
{\end{list}}



% para decidir si usar && o ^
\newcommand{\y}[0]{\ensuremath{\land}}

% macros de correctitud
\newcommand{\semanticComment}[2]{#1 \ensuremath{#2};}
\newcommand{\namedSemanticComment}[3]{#1 #2: \ensuremath{#3};}


\newcommand{\local}[1]{\semanticComment{local}{#1}}

\newcommand{\vale}[1]{\semanticComment{vale}{#1}}
\newcommand{\valeN}[2]{\namedSemanticComment{vale}{#1}{#2}}
\newcommand{\impl}[1]{\semanticComment{implica}{#1}}
\newcommand{\implN}[2]{\namedSemanticComment{implica}{#1}{#2}}
\newcommand{\estado}[1]{\semanticComment{estado}{#1}}

\newcommand{\invarianteCN}[2]{\namedSemanticComment{invariante}{#1}{#2}}
\newcommand{\invarianteC}[1]{\semanticComment{invariante}{#1}}
\newcommand{\varianteCN}[2]{\namedSemanticComment{variante}{#1}{#2}}
\newcommand{\varianteC}[1]{\semanticComment{variante}{#1}}

    
    \usepackage{a4wide}
    \usepackage{amsmath, amscd, amssymb, amsthm, latexsym}
    \usepackage[spanish,activeacute]{babel}
    \usepackage{enumerate}
    
    \setlength{\parskip}{0.1em}
    \usepackage{caratula} % Version modificada para usar las macros de algo1 de ~> https://github.com/bcardiff/dc-tex
    
    % -------- MACROS ----------
    
    % Define una tupla de enteros
    \newcommand{\tEnt}{\ent \times \ent}
    \newcommand{\bt}{\breakAndSpace{\tab}}
    
    % -------- END MACROS ------
    
    
    \begin{document}
    
    \titulo{TP de Especificación}
    \subtitulo{Juego de la vida toroidal}
    \fecha{\today}
    \materia{Algoritmos y Estructuras de Datos I}
    \grupo{Grupo: FrankerZ}
    
    % Pongan cuantos integrantes quieran
    \integrante{Manuel Panichelli}{072/18}{panicmanu@gmail.com}
    \integrante{Ignacio Alonso Rehor}{195/18}{arehor.ignacio@gmail.com}
    
    \maketitle
    
    \section{Problemas}
    %\aux{Aux}{i: \ent}{\bool}{\True}
    type $toroide = \matriz{\bool}$
    
    \subsection{esValido}
    % Ejercicio 1 : proc esValido(in t: toroide, out result : Bool)
    % Que dado un toroide verifique si es válido
    
    % Un toroide será válido si es una Matriz valida y no es vacío
    
    \begin{proc}{esValido}{
            \In t: $toroide$,
            \Out result: $\bool$}{}
        \pre{\True}
        \post{result = true \Iff esToroideValido(t)}
    \end{proc}
    
    % TODO: Mover a generales?
    % TODO: Que trucazo, no? ;)
    \pred{esMatriz}{m: \matriz{T}}{
        \paraTodo{i,j}{(0}{\longitud{m}} \implicaLuego \longitud{m[i]} = \longitud{m[j]})
    }
    
    \subsection{posicionesVivas}
    % Ejercicio 2 : proc posicionesVivas(in t: toroide, out vivas : seq<Z × Zi>)
    % Que dado un toroide devuelva todas las posiciones vivas.
    \begin{proc}{posicionesVivas}{
            \In t: $toroide$,
            \Out vivas: \TLista{\tEnt}}{}
        \pre{esToroideValido(t)}
        \post{sonPosicionesVivas(vivas, t) \wedge 
             (cantidadDePosicionesVivas(t) = \longitud{vivas})}
    \end{proc}
    
    \pred{sonPosicionesVivas}{vivas: \TLista{\tEnt}, t: $toroide$}{
        (\forall{p : \tEnt}) (p \in vivas \Then esPosicionViva(t, p))
    }
    
    \subsection{densidadPoblacion}
    % Ejercicio 3 : Que dado un toroide devuelva su densidad de poblaci ́on, es decir, la relaci ́on entre la cantidad de posiciones vivas y la cantidad total de posiciones.
    
    \begin{proc}{densidadPoblacion}{
            \In t: $toroide$, 
            \Out result: \float}{}
        \pre{esToroideValido(t)}
        \post{result \times cantidadTotalDePosiciones(t) = cantidadDePosicionesVivas(t)}
    \end{proc}
    
    \aux{cantidadTotalDePosiciones}{t: $toroide$}{\ent}{cols(t) \times rows(t)}
    
    \subsection{evolucionDePosicion}
    % Ejercicio 4 : proc evolucionDePosicion(in t: toroide, in posicion : Z × Z, out result : Bool)
    % Que dado un toroide y una posición del mismo, indique si dicha posici´on estar´ıa viva luego de un tick.
    \begin{proc}{evolucionDePosicion}{
            \In t: $toroide$,
            \In p: \tEnt,
            \Out result: \bool}{}
        \pre{esToroideValido(t) \wedge esPosicionValida(t, p)}
        \post{result = true \Iff viveLuegoDeUnTick(t, p)}
    \end{proc}
    
    
    \subsection{evolucionToroide}
    % Ejercicio 5 : proc evolucionToroide(inout t: toroide) Que dado un toroide lo evolucione un tick.
    \begin{proc}{evolucionToroide}{
            \Inout t: $toroide$}{}
        \pre{esToroideValido(t) \wedge t =$ T_{0}$}
        \post{esEvolucion(t, T_{0})}
    \end{proc}
    
    \subsection{evolucionMultiple}
    % Ejercicio 6 : proc evolucionMultiple(in t: toroide, in k: Z, out result: toroide)
    % Que dado un toroide t y un natural k, devuelva el toroide resultante de evolucionar t por k ticks.
    
    \begin{proc}{evolucionMultiple}{
            \In t : $toroide$, 
            \In k : \ent, 
            \Out result : $toroide$}{}
        \pre{(k \geq 1) \wedge
             esToroideValido(t)}
        \post{esK-EsimaEvolucion(result, t, k)}
    \end{proc}
    
    \subsection{esPeriodico}
    %Ejercicio 7: proc esPeriodico(in t: toroide, inout p: Z, out result: Bool)
    %Que dado un toroide devuelva si el mismo es peri ́odico o no. En caso de serlo, se debe devolver en la m ́ınima
    %cantidad de ticks en la cual se repite el patr ́on. Decimos que un toroide es peri ́odico si pasada cierta cantidad de ticks,
    % vuelve a tener exactamente la misma configuraci ́on que ten ́ıa originalmente.
    %Por ejemplo, en el siguiente caso el toroide es peri ́odico, y p = 2.
    
    \begin{proc}{esPeriodico}{
            \In t : $toroide$, 
            \Inout p: \ent,
            \Out result : \bool}{}
        \pre{esToroideValido(t)}
        \post{(result = \True \Iff tienePeriodicidad(t)) \wedge 
              esMinimaPeriodicidad(t, p)}
    \end{proc}
    
    \pred{tienePeriodicidad}{t: $toroide$}{
        (\exists{k: \ent})($tienePeriodicidadK-Esima(t, k)$)
    }
    
    % p es el minimo si:
    % - t tiene periodicidad p
    % - para todo q tal que t tiene periodicidad q-esima
    % => p <= q
    \pred{esMinimaPeriodicidad}{t: $toroide$, p: \ent}{
        $tienePeriodicidadK-Esima(t, p) $ \wedge \bt
        ((\forall{q: \ent})($tienePeriodicidadK-Esima(t, q) $ \implica p \leq q)
    }
    
    \pred{tienePeriodicidadK-Esima}{t: $toroide$, k: \ent}{
        (k \geq 1) \wedge ($esK-EsimaEvolucion(t, t, k)$)
    }
    
    \subsection{primosLejanos}
    %Ejercicio 8: proc primosLejanos(in t1: toroide, in t2: toroide, out primos: Bool)
    %Que dados dos toroides, devuelva si uno es la evoluci ́on m ́ultiple del otro.
    
    \begin{proc}{primosLejanos}{
            \In t1 : $toroide$, 
            \In t2 : $toroide$, 
            \Out primos : \bool}{}
        \pre{esToroideValido(t1) \wedge esToroideValido(t2)}
        \post{primos = \True \Iff sonPrimosLejanos(t1, t2)}
    \end{proc}
    
    % si son primos lejanos quiere decir que 
    % - t1 es alguna evolución de t2 o al revès
    % existe algun índice tal que t1 es K-esima evolución 
    \pred{sonPrimosLejanos}{t1: $toroide$, t2: $toroide$}{
        (\exists{k: \ent})
            ($esK-EsimaEvolucion(t1, t2, k) $ \vee
             $ esK-EsimaEvolucion(t2, t1, k)$)
    }
    
    \section{Predicados y Auxiliares generales}
    
    \subsection{Toroides}
    
    \pred{esToroideValido}{t: $toroide$}{
        esMatriz(t) \wedge \bt
        (rows(t) \geq 3) \wedge \bt
        (cols(t) \geq 3)
    }
    
    %/* Supone que es una posicion válida (no se va a ir de rango) /*
    \aux{valorEn}{t: $toroide$, p: \tEnt}{\bool}{
        t[p_{0}][p_{1}]
    }
    
    \pred{esPosicionViva}{t: $toroide$, p: \tEnt}{
        % Chequea que la pos se vayan de rango i,j
        esPosicionValida(t, p) \yLuego
        % Chequea que t[i,j] sea una pos viva.
        (valorEn(t, p) = \True)
    }
    
    \aux{cantidadDePosicionesVivas}{t: $toroide$}{\ent} {
        cantidadDeAparicionesEnMat(True, t)
    }
    
    \pred{esEvolucion}{t, t_{0}: toroide}{
        (esToroideValido(t) \wedge esToroideValido(t_{0})) \wedge \bt
        (mismaDimension(t, t_{0}))\wedge \bt
        (\forall{p: \tEnt}) (esPosicionValida(t, p) \implicaLuego valorEn(t, p) =  viveLuegoDeUnTick(t_{0}, p))
    }
    
    \pred{mismaDimesion}{t, t_{0}: toroide}{
        (cols(t) = cols(t_{0})) \wedge \bt
        (rows(t) = rows(t_{0}))
    }
    
    
    \pred{esK-EsimaEvolucion}{t, t_{0}: toroide, k: \ent}{
        (\exists{ts: \TLista{toroide}})( \bt
            %$\text{/* t_{0} es la primera}$
            (ts[0] = t_{0}) \wedge
            %\text{/* t es la ultima evolución */}
            (ts[k] = t) \wedge \bt
            %\text{/* Todo elemento y su consecutivo cumple que uno es evolución del otro */}
            (\paraTodo{i}{(0}{\longitud{ts} - 1}) \implicaLuego 
                (esEvolucion(ts[i + 1], ts[i])
        )
    }
    
    \subsubsection{Vecinos}
    % - Cualquier posici´on viva con menos de 2 vecinas vivas, muere (por soledad)
    % - Cualquier posici´on viva con 2 o 3 vecinos vivos, vive.
    % - Cualquier posici´on viva con m´as de 3 vecinas vivas, muere (por %superpoblaci´on)
    % - Cualqueir posici´on muerta con exactamente 3 vecinos vivos, pasa a vivir (por reproducci´on)
    
    \pred{viveLuegoDeUnTick}{t: $toroide$, p: \tEnt}{
        (\neg esSoledad(t, p) $ \wedge$ \neg esSuperpoblacion(t, p)) \wedge \bt
        (esSupervivencia(t, p) \vee esReproduccion(t, p))
    }
    
    \pred{esSoledad}{t: $toroide$, p: \tEnt}{
        esPosicionViva(t, p) \wedge
        (cantidadVecinosVivos(t, p) < 2)
    }
    
    \pred{esSuperpoblacion}{t: $toroide$, p: \tEnt}{
        esPosicionViva(t, p) \wedge
        (cantidadVecinosVivos(t, p) > 3)
    }
    
    \pred{esSupervivencia}{t: $toroide$, p: \tEnt}{
        esPosicionViva(t, p) \wedge
        (2 \leq cantidadVecinosVivos(t, p) \leq 3)
    }
    
    \pred{esReproduccion}{t: $toroide$, p: \tEnt}{
        ($\neg$ esPosicionViva(t, p)) \wedge
        (cantidadVecinosVivos(t, p) = 3)
    }
    
    \aux{cantidadVecinosVivos}{t: $toroide$, p: \tEnt}{\tEnt}{
        \sum_{i=-1}^1 \sum_{j=-1}^1 \IfThenElse{(i, j) = (0,0)}{0}{(
            \IfThenElse{esPosicionViva(t, trasladar(t, p, (i, j))}{1}{0}
        )}
    }
    
    % TODO: Hacer lindo
    % TODO: Preguntar si la funcion "dada" mod cumple que el resultado es mayor o igual a 0
    \aux{trasladar}{t: $toroide$, p: \tEnt, d: \tEnt}{\tEnt}{
        (mod(p_{0} + d_{0}, cols(t), \bt
         mod(p_{1} + d_{1}, rows(t))
    }
    
    \subsection{Matrices}
    
    %/* Para filas y columnas suponemos que m es una matriz */\\
    \aux{cols}{m: \matriz{T}}{\ent}{\longitud{m}}
    \aux{rows}{m: \matriz{T}}{\ent}{\IfThenElse{cols(m) > 0}{\longitud{m[0]}}{0}}
    
    \aux{cantidadDeAparicionesEnMat}{x: T, m: \matriz{T}}{\ent}{
        \sum_{i=0}^{cols(m) - 1}
        \sum_{j=0}^{rows(m) - 1}
        (\IfThenElse{s[i][j] = x}{1}{0})
    }
    
    \pred{esPosicionValida}{m: \matriz{T}, p: \tEnt}{
        % Chequea que ambos estén en rango
        (0 \leq p_{0} < cols(m)) \wedge \bt
        (0 \leq p_{1} < rows(m))
    }
    
    \section{Decisiones tomadas}
    % Sólo para decisiones de alto nivel, no para explicar la solución a los ejercicios: CONSULTAR!
    - Tomamos al Toroide como una Matriz compuesta por una \TLista{\TLista{\bool}}
    Donde la primera representa a las columnas, y la segunda a las filas.
    Accederemos a ella mediante tuplas $(x, y) : \tEnt$, donde $x$ es la posición en la columna e $y$ en la fila.
    
    - Tomamos como decisión que un toroide válido será aquel que sea al menos de $3 \times 3$, de forma tal que los 8 vecinos de cualquier posicion sean siempre diferentes entre sí.
    
    \end{document}
    
    